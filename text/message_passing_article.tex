\documentclass[aps, prl, twocolumn, a4paper, floatfix]{revtex4}
%\documentclass[twocolumn]{revtex4}
\usepackage{graphicx}
\usepackage{amsmath}
\usepackage{placeins}
\usepackage{color}
\usepackage{hyperref}
%\usepackage{bug}
%\bibliographystyle{plain}
\newcommand{\er}{Erd\H{o}s-R\'{e}nyi}
\newcommand{\kk}{\langle k \rangle}



\begin{document}

\title{Secure message passing on networks with insecure nodes}
\author{Sebastian M.\ Krause}
\affiliation{Ruder Boskovic Institute, Zagreb, Croatia}
\author{Vinko Zlatic}
\affiliation{Ruder Boskovic Institute, Zagreb, Croatia}
\author{Michael M.\ Danziger}
\affiliation{Deptartment of Physics, Bar Ilan University, Ramat Gan, Israel}


\begin{abstract}
It is often necessary to transmit a message across a network when parts of the network are not secure.
Here, we consider the case of a network partitioned into sets of nodes with the assumption that no single subset can be trusted.
As such, the message needs to be divided and transmitted on multiple paths so that no subset sees the entire message.
This problem arises, for instance, in a p2p network running different unpatched software versions 
and when considering AS-level listeners in the entry and exit to the Tor network.

We present a general analysis of this problem on random graphs including analytic solutions for \er and scale-free networks and numerical simulations confirming our calculations and further numerical tests on real-world networks including the internet, partitioned by AS.
\end{abstract}
%\pacs{89.65.-s, 05.50.+q, 05.65.+b, 64.60.De}
\maketitle

Secure and anonymous communication over networks, in particular the internet, has become a central question facing the global community.
In light of widespread state-surveillance, large-scale cybercrime and superbugs like ``Heartbleed,'' one can no longer assume that an entire communications network is secure.
However, the network insecurity may be disjoint.  
For instance, on a p2p network, there may be different versions of the software running at the same time.  
In such a case, though there may be unpatched or even undiscovered bugs affecting a given version, it is unlikely that all of the versions will be compromised by the same group at the same time.
A sensible strategy for securing the communication would then be to divide the message and transmit it in such a way that \textit{no single version} receives the entire message.
This problem arises when attempting to safeguard anonymity with the Tor network.  
Recent work has shown that if the same autonomous system (AS) controls a router on the path from the source to the entry node of the Tor network and also a router from the exit node to the destination, the anonymity of the Tor network is broken.
Since a relatively small number of AS's control the entire internet, this scenario is a serious concern.

Here we examine this problem on general network topologies.  
We begin with a formal definition of the problem and its relationship to percolation theory and then proceed to demonstrate a number of key properties on random graphs and sample measurements on real-world networks, including the AS-level internet.

To analyze this problem, we consider a network for which each node is assigned exactly one color.
A pair of nodes is color connectable iff there exist a set of paths connecting the nodes such that no color appears on all of the paths.  


When passing a message through a network, there may be sets of insecure nodes.
Assuming the message can be split and that the message can only be reconstructed from the entire 

Many networks are naturally partitioned into different sets which are not 

When passing information across a network, the network which is partitioned into sets, each one of which is considered insecure.


\end{document}